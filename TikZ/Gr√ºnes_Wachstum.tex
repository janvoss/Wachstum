\documentclass[border=10pt]{standalone}
\usepackage{smartdiagram} %für tikz
\usepackage[ngerman]{babel} %Sprache deutsch (für Silbentrennung)
\usepackage[utf8]{inputenc} %UTF-Code für Umlaute
\usepackage[T1]{fontenc} %Trennung von Wörtern mit Umlauten
\usepackage{lmodern} %Schriftart
%\usepackage{libertine}
\renewcommand*\familydefault{\sfdefault} %Serifenlose Schrift als Standard
\usepackage{microtype}
\usesmartdiagramlibrary{additions}
\usetikzlibrary{fit}
\usetikzlibrary{decorations.pathreplacing}

\tikzstyle{container} = [draw, rectangle, semithick, inner sep=0.3cm
]

%Aufzählungsstriche
\AtBeginDocument{
	\def\labelitemi{\normalfont\bfseries{--}}
}


\begin{document}

		\begin{tikzpicture}[
		every node/.style = {shape=rectangle, % is not necessary, default node's shape is rectangle
			rounded corners,
		%	draw, semithick,
			text width=7cm,
			align=center,
			node distance=0.1cm
		}
		]
		
%Überschriften
	\node (Pro)[text depth=.75ex  % Schrift immer auf der gleichen Höhe
   		]{\textbf{Grünes Wachstum ist möglich}
     };
    
    \node (Contra)[text depth=.75ex, right =  of Pro
    ]{\textbf{Grünes Wachstum ist unmöglich}
    };



%Spalten
	\node[below= of Pro, align=left](){
	\begin{itemize}
		\item Erfolg erneuerbarer Energien ist ein gutes Beispiel
		
		\item Es gibt doch Beispiele! China wird z.B. grüner!
		
		\item Fondsgesellschaften legen grüne Finanzprodukte auf
		
	\end{itemize}
	};

\node[below= of Contra, align=left](){
	\begin{itemize}

		
		\item Historische Erfahrung spricht dagegen
		
		\item Lobbyismus spricht dagegen
		
		\item Kapitalverwertungsinteresse großer Investoren
		
		\item auch grünes Wachstum braucht Ressourcen
		
		\item Es fehlt der politische Wille
		
	\end{itemize}
};


		
     
%Horizontale Linie unter Überschriften	
	\draw [
	transform canvas={yshift=-0.1cm}
	] (Pro.south west) -- (Contra.south east);
	

	
%Überschrift
	\path [draw=none] (Pro) -- (Contra) node [midway] (Mittelpunkt) {};	
	\node[above= 1cm of Mittelpunkt, text width=15cm] (Ueberschrift) {\huge Grünes Wachstum?};
	
	\node[below=of Ueberschrift] () {11. Juni 2024};
	
	
	
	
	
	\end{tikzpicture}

\end{document}